\documentclass[12pt]{article}


\usepackage[authoryear]{natbib}
\usepackage{amsmath}
\usepackage{hyperref}
\usepackage{hyperref}
\usepackage{geometry}
\usepackage{graphicx}
\usepackage{amsfonts}

\input{../../paperProduction/occBind/docs/AMArepresentationNewCmds}



\author{Gary S. Anderson\thanks{The analysis and conclusions set forth are those of the author and do not indicate concurrence by other members of the research staff or the Board of Governors. I would like to thank Luca Guerrieri, Christopher Gust, Hess Chung, Benjamin Johannsen  and Robert Tetlow for their comments and suggestions.  Special thanks to Luca Guerieri for first noticing that the series representation formulation could lead to an error bound for model solutions.}}

\title{A New Series Representation for Time Invariant Maps that
 Arise in  Nonlinear Dynamic Stochastic Economic Models}


\begin{document}

\maketitle

\begin{abstract}
This paper proposes a new series representation of solutions
for dynamic stochastic models and develops an algorithm for computing
 time invariant discrete time maps that accurately characterize
the solutions for a wide array of nonlinear rational expectations models. 
The series representation makes it possible to reliably improve upon an initial
guess for a stochastic model decision rule by solving a 
potentially complicated, but deterministic problem in the initial time period.
The algorithm can handle models with occasionally binding constraints and/or regime switching. 
\newpage
\tableofcontents
\newpage
The series representation also provides a formula for computing 
error bounds for any proposed time invariant model solution.
These error bounds should prove useful for assessing the accuracy of 
proposed model solutions independent of the algorithmic source.




This paper proposes a new series representation of solutions
for dynamic stochastic models and develops an algorithm for computing
 time invariant discrete time maps that accurately characterize
the solutions for a wide array of nonlinear rational expectations models. 
The series representation makes it possible to reliably improve upon an initial
guess for a stochastic model decision rule by solving a 
potentially complicated, but deterministic problem in the initial time period.
The algorithm can handle models with 
occassionally binding constraints and/or regime switching. 


The series representation also provides a formula for computing 
error bounds for any proposed time invariant model solution.
These error bounds should prove useful for assessing the accuracy of any
proposed model solution regardless of the source.



Anderson has developed a new series representation useful for
solving a wide class of nonlinear dynamic stochastic models.
The solutions computed by the technique accommodate the possibility that 
model trajectories can depart from and re-engage
occasionally binding constraints as well as transition between various regimes. Remarkably, this series representation makes it possible to characterize model solutions as a linear sum of orthogonal functions.  



  I believe I have the only algorithm that provides useful error bound on the errors in the decision rule based on readily available calculations with the original nonlinear model.  These calculations should be useful to modellers whether or not they use other components of my algorithm. The algorithm is designed to exploit parallelism and I expect that the technique will scale up very nicely.


This paper proposes a new series representation for bounded
time invariant discrete time maps.
The paper uses the representation to develop  formulae
for computing accuracy bounds for any proposed time invariant model solution.
The paper also provides an algorithm for computing
solutions for dynamic economic models.

With the series representation for time invariant maps in hand,
the paper shows how to use the solution to a deterministic problem at
time t, to improve upon a proposed solutions for
any given dynamic economic models with occasionally
binding constraints or regime switching or both.   The error bounds from the
series representation help with determining algorithmic convergence by
characterizing the potential benefits
of further solution refinement.

This paper shows how to apply the formulae in\citep{anderson10} to compute 
rational expectations solutions for models linear except for occasionally binding constraints.  The formulae facilitate the recursive 
computation of solutions that honor the 
constraints for successively 
longer horizons. The solutions thus computed
accommodate the possibility that model
trajectories may depart from and re-engage the constraints.
The technique is applicable for nonlinear inequality constraints.



This paper uses a series representation for bounded solutions
to dynamic models to compute
  time invariant discrete time
maps that accurately characterize
the solutions for a wide array of nonlinear
rational expectations models. This solution also provides a formula for computing accuracy bounds for any proposed time invariant model solution.
Support vector machine function approximation provides
error bounds that 
are especially useful used in conjuction with the series representation
error formulae.
One can  dynamically adjust the representation accuracy
as needed to guarantee convergence to a true solution.



This presentation reports on the efficiency gains associated with exploiting this linearity and with parallelizing the
computation of the function approximations.

Support vector machine function approximation provides
error bounds that 
are especially useful used in conjuction with the series representation
error formulae.
One can  dynamically adjust the representation accuracy
as needed to guarantee convergence to a true solution.



\end{abstract}


Support vector machines occupy a prominent place applied machine learning
Additionally they can reduce the burden of computation as they determine 
a subset of points that are important in characterizing a given function.
Kernel trick,  powerful representation quadratic programming solution to compute there are on-line techniques for adding and deleting individual points from the representation.  A weighted sum of kernel functions RBF, wavelet, special forms for time series.
  This paper proposes a new series representation for bounded so-
lutions to dynamic models. This series representation can be used
to determine a series representation for time invariant discrete time
maps that characterize the solutions to many models. Consequently,
the technique constitutes an important component in a technique for
accurately characterizing the solutions for a wide array of nonlinear
rational expectations models. It can also provide a formula for com-
puting accuracy bounds for any proposed time invariant model so-
lution. The series representation serves as an important component
in an algorithm for constructing approximate solutions for nonlinear
rational expectations models.
The technique recursively computes solutions that honor the con-
straints for successively longer horizons. The technique also provides
a metric for determining apriori how long the horizon must be for a
given level of accuracy of the state vector. The solutions computed by
the technique accommodate the possibility that model trajectories can
depart from and re-engage inequality constraints as well as transition
between various regimes.
This paper applies support vector machine function approximation to
reduce the computational burden associated with representing the
unknown, potentially highly nonlinear stochstic functions that arise in
solving dynamic models with occasionally binding constraints.


Support vector machines 
have become an essential tool in contemporary machine learning research
where computer scientists exploit their flexibility and
computational tractability in modelling complex high dimensional data.
Like many other function approximation approaches,
support vector machines represent functions as a linearly weighted sum
of a family of basis functions.  They differ from other approaches in  the
use of ``hinge loss functions'' that generate
an easy to solve
quadratic programming problem(QPP) for determining the weights.
The solution of this QPP identifies a subset of points, the support vectors,
that are influential in the representation.  With strategically chosen
basis functions, this can dramatically reduce the number of terms needed
to approximate a function to a given level of accuracy.


\section{Introduction and Summary}





Stochastic dynamic non linear economic
models increasingly embody  occasionally binding constraints (OBC) essential
Since \cite{Christiano2000} a host of
authors have described a variety of approaches. 
\cite{holden15:_exist_dsge,guerrieri15:_occbin,benigno09,hintermaier10,brumm10,nakov08,haefke98,nakata12,gordon11,billi11,Hintermaier2010,Guerrieri2015}
This paper provides yet another.  This new series representation provides  a coherent framework for attacking a wide variety of complicated nonlinear models.
The framework provides a new way to bound the error one can expect from
employing a given proposed model solution and leads to a
algorithm with  components similar to parameterized expectations that
one can use to improve proposed solutions. The series representation makes
it possible to organize the calculation around computing a deterministic
problem at time t given a proposed solution.  The deterministic solution
can accommodate inequality constraints or alternative regimes to produce a
solution for each set of initial conditions.  One can typically arrange,
perhaps by adding auxiliary variables, to produce a ``decision rule''
that one can correctly integrate to get a deterministic conditional
expectation function that can be iterated forward and serves to
improve upon the original proposed solution.
Time invariant stochastic functions 
lead naturally to an associated family of deterministic maps
which can conveniently represented by the series representation.





\section{A New Series Representation For Nonlinear Dynamic Stochastic Time Invariant Maps}
\label{sec:newseries}

\subsection{A Linear Reference Model and a Formula for  ``Anticipated Shocks''}
\label{sec:linref}
\begin{itemize}
\item Almost Arbitrary
\item Impact of Choice
\end{itemize}


\subsection{A Series Representation for Bounded Paths}
\label{sec:boundedpaths}



\subsubsection{Assessing $x_t$ Error}
\label{sec:truncationerr}
\paragraph{Truncation Error}
\paragraph{Pessimistic Error}

\subsubsection{A Path Norm}
\label{sec:pathnorm}

\subsection{Application to Time Invariant Maps}
\label{sec:extToMaps}


\subsubsection{Dynamic Stochastic Models, Model Solutions  and Conditional Expectations}
\label{sec:condExp}

\subsubsection{An RBC Model Example}
\label{sec:rbcexample}


\subsubsection{Relation to Impulse Response Functions and Stochastic Simulations}
\label{sec:relImp}


Use (Impulse Response Functions?)

Proceeding as though no uncertainty in the parameters.
\href{https://books.google.co.uk/books?id=kLiRDgAAQBAJ&pg=PA77&lpg=PA77&dq=nonlinear+models+impulse+response+function+conditional+expectations&source=bl&ots=1pCjlcgxSu&sig=WA9INrvXVcFhVOd_2wq8-V6alw8&hl=en&sa=X&ved=0ahUKEwid1Y3fvuPWAhUlKsAKHT7lCDkQ6AEITjAF#v=onepage&q=nonlinear%20models%20impulse%20response%20function%20conditional%20expectations&f=false}{book on nonlinear impulse response}
  Potter 1995 2000 and Koop 1996

  from  Elements of Nonlinear Time Series Analysis and Forecasting  Jan G. De Gooijer

CONDITIONAL EXPECTATIONS ASSOCIATED WITH STOCHASTIC PROCESSES
R . A. BROOKS


\href{http://www.pnas.org/content/103/11/3968.full}{discrete dynamical conditional expectation} 

\subsection{Computing Model Solution Error Bounds}
\label{sec:solnerrorbounds}

\subsubsection{Worst Path: Pessimistic Error Bound}
\label{worst}

\subsubsection{Practical Considerations for Applying the Formula}
\label{sec:practicalformula}


\paragraph{MSNTO}
\paragraph{Assessing a Proposed Soluion}
\paragraph{RBC Example for Known Solution}
\paragraph{RBC Example for Unknown Solution}
\paragraph{Algorithm Agnostic: Your Solution Here}

\section{An Algorithm for Improving Proposed Time Invariant Solutions}
\label{sec:algoforsoln}

\subsection{Auxiliary Variable Reformulation}
\label{sec:aux}

\subsubsection{RBC Example}
\label{sec:rbcaux}

\subsubsection{General Formulation}
\label{sec:genAux}

\subsection{Function Approximation Representation}
\label{sec:funcApproxRep}

\subsubsection{General Issues}
\label{sec:generalissues}

\paragraph{Function composition}



\paragraph{Correct Expectations}

\paragraph{Precomputig Integrals}


\subsubsection{Smolyak Interpolation}
\label{sec:smolyakinterp}

\paragraph{Anisotropic}

\paragraph{Ergodic Set}

\subsection{Algorithm Overview}
\label{sec:algoverview}


\begin{figure}
  \centering
  


  \begin{gather}
    \fbox{Nonlinear Rational Expectations Model}\\ \Downarrow\\
\fbox{Bounded Time Invariant Function Solution}\\\Downarrow\\
\fbox{Series Representation}\\\Downarrow\\
\fbox{Series Approximation}
  \end{gather}
  \caption{From Models to Approximate Solutions}
  \label{fig:modelsto}
\end{figure}



This series representation for deterministic maps,
broadly applicable for nonlinear rational expectations models 
leads to a formulae for accuracy bounds for any proposed solution.
As an important component in an algorithm for
constructing approximate solutions it
facilitates exploiting recursive computation of the solutions for complicated models.
The algorithm has been implemented in
Mathematica code that can compute solutions for
nonlinear models with occasionally binding constraints and/or regime switching models.






\subsubsection{Generality}
\label{sec:generality}




\begin{itemize}
\item Time Invariant Maps
\item Bounded Solutions
\item Exists a Representation Problem is to Find One
\item Algorithm terminates with a Proposed Solution With Solution
  No further away from a true soltion than the specified tolerance 
\item No Existence Guarantees or Uniquess Guarantees.
\item Parallizable
\end{itemize}


Consider a family of {\it time invariant } stochastic functions:
 \begin{gather}
   \xWarg \in{R^L}\,\text{ with }\,\infNorm{\xWarg}  \le \bar{\mathcal{X}}\,\,\forall t> 0 \label{fFamily}.
 \end{gather}
The $x_{-1}$ is an  $L$ dimensional state vector and $\epsilon$ is a $K$ dimensional ``shock'' vector that together index
individual trajectories for future state vectors.  
Each member of the family characterizes the evolution of a {\em deterministic} trajectory of values.\footnote{Subsequent sections describe how these deterministic trajectories are useful for representing a wide array stochastic model solutions.}

It will prove useful to also define a time invariant deterministic function $\XtFuncTI\equiv \expctEps{\xtFuncTI}$ and denote
\begin{gather*}
\xsubtFunc{t+k}{(x_{t-1},\epsilon_t)}\equiv\begin{cases}
\xtFunc{(x_{t-1},\epsilon_t)} &k=0\\
\XtFunc{(\xsubtFunc{t+k-1}{(x_{t-1},\epsilon_t)})} &k>0
\end{cases}
\end{gather*}


\paragraph{Proposed Conditional Expection}

\paragraph{Deterministic Problem}

\paragraph{Conditional Expectation Update}

\subsubsection{Algorithm Pseudocode}
\label{sec:pseudocode}

\subsubsection{RBC Example}
\label{sec:generalRBCExample}

\paragraph{unknown solutions}
\begin{description}
\item[Solution]
\item[Error Bound]
\end{description}

\paragraph{unknown solutions occasionally binding constraints}
\begin{description}
\item[Solution]
\item[Error Bound]
\item[resources]\
  \begin{itemize}
  \item 
  \end{itemize}
\end{description}


\subsection{ Other Examples}
\label{sec:otherexamples}



\section{Future Work}
\label{sec:future}

\begin{itemize}
\item time varying:  regime switching, hammers need nails
\item SVMR and other ML universal approximators
\item function composition Fa Di Bruno
\item divide and conquer solution space initially collect results stocastic steady state at center of divided region with overlap
\item dynamical system theory implications
\item relation to Koopman Operators
\item Implications of almost arbitrary.  Small F eigenvalues versus small z values
\end{itemize}

\section{Conclusions}
\label{sec:conc}




\bibliographystyle{plainnat}
\bibliography{anderson,files}

\end{document}

