\documentclass[hyperref]{labbook}
\usepackage{moreverb}
\usepackage{natbib}
\usepackage{hyperref}
\usepackage{graphicx}

\begin{document}

\frontmatter
\title{AMA Series}
\author{Gary Anderson }
\maketitle

\printindex
\tableofcontents




\mainmatter
\begin{itemize}
\item \href{https://orgmode.org/manual/Structure-editing.html}{org editing}
\item \href{http://ergoemacs.org/emacs/emacs_package_system.html}{emacs package system}
\item org-meta-return doesn't seem to be defined
\end{itemize}



\labday{Wednesday May 30, 2018}

\begin{itemize}
\item convergence problems when using delta in collocation solutions occurs at specific collocation points
\item probably conditional expectation at those points goes out of the grid
\includegraphics{duh.pdf}
\item need to distinguish between divergent and non convergent solutions
\item should monitor evolution of the solution for the worst few.  if just bouncing around marginally time to stop.  if consistently moving away should stop
\item probably just do time series regression and test for null of zero coefficients
\end{itemize}


\labday{Monday May 28, 2018}

\begin{itemize}
\item \href{ssh ublm1 /var/appl/matlmd/scripts/mathlm status}{math icence status}
\end{itemize}
\labday{Monday May 21, 2018}

\begin{itemize}
\item \href{https://www.blazemeter.com/blog/how-start-working-github-plugin-jenkins}{jenkins on mac setup}
\item \href{https://dev.to/iriskatastic/start-continuous-integration-with-jenkins-pipeline-4edb}{jenkins and CI}
\end{itemize}
\labday{Sunday May 20, 2018}

\begin{itemize}
\item \href{https://mathematica.stackexchange.com/questions/163912/automating-testing-of-mathematica-code}{setting up continuous integration testing with mathematica}
\item \href{https://medium.com/@briananderson2209/best-automation-testing-tools-for-2018-top-10-reviews-8a4a19f664d2}{best test automation, most not useful except for web apps}
\item \href{https://testanything.org/producers.html}{test-anything-protocol producers}
\item \href{https://testanything.org/consumers.html}{tap consumers}
\end{itemize}


\labday{Friday May 18, 2018}
\begin{itemize}
\item simplify model spec to just ergodic parallel regime format
\end{itemize}


\begin{itemize}
\item 

\begin{verbatim}
{ docker run   -u root   --rm   -d   -p 8080:8080  -p 50000:50000  -v jenkins-data:/var/jenkins_home   -v /var/run/docker.sock:/var/run/docker.sock   jenkinsci/blueocean }{jenkins installation}

\end{verbatim}
\end{itemize}

\labday{Thursday April 21, 2018}

\begin{itemize}
\item \href{https://atom.io/packages/linter-mathematica}{Atom linter for mathematica}
\end{itemize}

\labday{Thursday March 26, 2018}
\begin{itemize}
\item approx 6,6 singular matrix  trying 6,3
\end{itemize}
\labday{Monday March 26, 2018}

\begin{itemize}
\item List,of xvars for initializing different triples
\item pre and post perhaps better way to do bang bang control
\item perhaps series can handle models that don't explode too fast
\item should use pre post final to fail unconstrained non optimality
\end{itemize}
\labday{Sunday March 25, 2018}

\begin{itemize}
\item \href{http://people.clarkson.edu/~mbudisic/software.html}{mesochronic code}
\end{itemize}
\labday{Wednesday March 21, 2018}

  \begin{itemize}
  \item \href{https://www.wolfram.com/company/careers/opportunities/#op-94325-software-developer-in-test}{CI Test job at Wolfram}
  \item \href{https://github.com/lynchs61/Mathematica-Test-Runner}{math test runner}
  \item \href{https://testanything.org/consumers.html}{test anything protocol: tap consumers}
  \item \href{http://reference.wolfram.com/language/tutorial/UsingTheTestingFramework.html}{mathematica testing getting started}
  \item \href{https://wiki.jenkins.io/display/JENKINS/TAP+Plugin}{jenkins tap consumer }
  \item \href{https://www.atlassian.com/software/jira}{project tracking}
  \end{itemize}


\labday{Sunday February 23, 2018}
\experiment{Check performance of Complementary Slackness version}

\begin{itemize}
\item 
\end{itemize}
\labday{Friday February 23, 2018}

\begin{itemize}
\item use classifier to characterize ergodic set  MSNTO pick points SVM characterize region
\end{itemize}

\labday{Tuesday February 13, 2018}

\begin{itemize}
\item \href{http://www.runmycode.org/faq.html}{runmycode}
\item \href{http://juliacon.org/2018/cfp}{julia con}
\end{itemize}

\labday{Friday February 9, 2018}

\begin{description}
\item[doItSeries] \ 
  \begin{itemize}
  \item (1,1,1),1,1
%  \includegraphics{resDir/forBetterRBC--1,1,1-Iters30theK0host-MacBook-Pro-2numKern2ApproxTrad.pdf}
  \end{itemize}
\end{description}

\labday{Friday January 28, 2018}

\begin{itemize}
\item \href{https://docs.julialang.org/en/stable/manual/performance-tips/}{julia performance tips}
\item \href{https://github.com/JuliaParallel/ClusterManagers.jl}{clustermanagers}
\item \href{https://groups.google.com/forum/?fromgroups=#!topic/julia-dev/21AGMrqbuM0}{julia functional programming}
\end{itemize}

\labday{Tuesday Januaryt 23, 2018}


\begin{itemize}
\item {http://reference.wolfram.com/language/tutorial/ConstrainedOptimizationLocalNumerical.html}{math numerical optimization constraints}
\item \href{http://reference.wolfram.com/language/tutorial/ConstrainedOptimizationLocalNumerical.html}{math numerical optimization constraints} from google ``mathematica nonlinearinteriorpoint''
\item \href{http://www.johnboccio.com/MathematicaTutorials/08_ConstrainedOptimization.pdf}{Wolfram tuitorial constrained optimization}
\item \href{https://lectures.quantecon.org/jl/}{QuantEcon lecures julia}
\end{itemize}



\labday{Monday Januaryt 22, 2018}


\begin{itemize}
\item \href{https://github.com/JuliaInterop/Mathematica.jl}{Julia call Mathematica}
\item \href{https://github.com/JuliaParallel/ClusterManagers.jl}{julia slurm}
\item \href{http://ucidatascienceinitiative.github.io/IntroToJulia/Html/HPCJulia}{HPC Julia}
\item \href{http://ucidatascienceinitiative.github.io/IntroToJulia/Html/HPCJulia}{Julia downloads}
\item \href{http://reference.wolfram.com/language/tutorial/ConstrainedOptimizationGlobalNumerical.html}{Mathematica optimization documentation}
\item \href{https://lectures.quantecon.org/jl/getting_started.html}{Julia Quantitative Economics}
\item \href{https://docs.julialang.org/en/stable/manual/embedding/}{calling julia from c fortran}
\end{itemize}

\labday{Sunday December 10, 2017}

\begin{itemize}
\item 01 no errors
\item Xlib:  extension "MIT-SHM" missing on display "localhost:10.0" repeatedly
Think related to quitting with control backslash  I tried  control C then exit to linkread didn't work, but backout of linkread worked  ps -aef had many fewer mathematica related process after backout than exit
  \begin{itemize}
  \item using kkSSNow to sett initial guess for 01 no errors 01 and mean 3.78,.99
  \item seems to work for notbinding alone up to 4*{1,1,1}
  \item not distributing work broadly all on kernel 15 or 10 
  \end{itemize}

\end{itemize}


\labday{Saturday December 9, 2017}

\begin{itemize}
\item code seems to switch back and forth between using multiple kernels and a single kernel 15
\end{itemize}


\labday{Thursday December 8, 2017}

\begin{itemize}
\item betterRBCCompSlack
  \begin{itemize}
  \item try notbinding always true
  \item then try binding always true
  \item compare with occ bind online solutions
  \end{itemize}
\end{itemize}

\labday{Thursday December 7, 2017}

\begin{itemize}
\item betterRBCCompSlack
  \begin{itemize}
  \item try notbinding always true
  \end{itemize}
\end{itemize}

\labday{Saturday November 21, 2017}
\begin{itemize}
\item problems with MSNTO  functions differ accross kernels when evaluating
\end{itemize}



\labday{Saturday November 4, 2017}

\begin{itemize}
\item VAR expectations in FRBUS
\item feedback from Chris after Tuesday November 7
\item paper notes post seminar
  \begin{itemize}
  \item event monitors 
``probability of hitting lower bound staying two periods''
  \item Smets model
  \item FRBUS
  \item measure parallel speedup
  \item fix broken code solve RBC examples and provide error boounds
  \item verify using ergodic works
  \item polynomial composition Fa Di Bruno
  \item example random series with inequality constraint
  \end{itemize}

\end{itemize}



\begin{itemize}
\item do linearsolve in parallel
\item xkzk func chose negative capital compiled function failed perhaps just need to make sure number of series terms adequate. happened with only two terms but high accuracy for anisotropic
\item still need progress monitor
\item does ergodic work?
\item should compute error bounds actual error versus series length and number of recursions (new conditional expectations) exploit an-isotropic
\item increase Mma precisoion to try higher degree approximation 
\item some files
\begin{verbatim}
-rw-r--r-- 1 m1gsa00 msu   1180 Oct 30 16:51 someParallel.mth
for sbatch

-rw-r--r-- 1 m1gsa00 msu   1926 Oct 30 17:03 AMAFedsBetterRBC.mth
defines doIt

-rw-r--r-- 1 m1gsa00 msu    827 Oct 31 11:11 basicRBC.mth
for debugging code with betterRBC.m

-rw-r--r-- 1 m1gsa00 msu    343 Oct 31 13:52 interParallel.mth
interactive parallel slow initialization runs [{4,4,2},6,3]
fails with huge numbers 

-rw-r--r-- 1 m1gsa00 msu  10261 Oct 31 13:52 cleanTrans.mth
fails cmpXValsfails to evaluate

-rw-r--r-- 1 m1gsa00 msu    687 Nov  4 14:05 arbGenPath.mth
-rw-r--r-- 1 m1gsa00 msu   4723 Nov  4 14:07 genArbLin.mth 
generate graphs for paper
\end{verbatim}
  \item Chris suggests focus on describing context in literature
\item VAR expectations in FRBUS
\item feedback from Chris after Tuesday November 7
\item paper notes post seminar
  \begin{itemize}
  \item event monitors 
``probability of hitting lower bound staying two periods''
  \item Smets model
  \item FRBUS
  \item measure parallel speedup
  \item fix broken code solve RBC examples and provide error boounds
  \item verify using ergodic works
  \item polynomial composition Fa Di Bruno
  \item example random series with inequality constraint
  \item presentation notes
    \begin{itemize}
    \item my models with boolean gates somewhat confusing unintuitive.
neokeynesian NK model no lagrangian
\item what happens for models with no solution or multiple solutions
\item perhaps better not to emphasize new solution but that useful in any soluton algorithm  just better parameterized expectations
\item Ben would like to recode and solve NK model with finite elements interested in E-Stability 
    \end{itemize}
  \end{itemize}

\end{itemize}

\bibliographystyle{plainnat}
\bibliography{emds.bib}
\end{document}
