%\documentclass[tikz]{beamer}
\documentclass[notheorems]{beamer}

%\usepackage{tikz}
%\usetikzlibrary{shapes,arrows}
\mode<presentation>{}
\usepackage{beamerthemeshadow}
%\let\definition\relax
\let\theorem\relax
\input{../../paperProduction/occBind/docs/AMArepresentationNewCmds}


\author{Gary S. Anderson\thanks{The analysis and conclusions set forth are those of the author and do not indicate concurrence by other members of the research staff or the Board of Governors.}}

\title{A New Series Representation for 
Nonlinear Dynamic Stochastic Model Solutions: General Error Bounds and a New Solution Algorithm}

\date{\today: \currenttime}
\begin{document}


\begin{frame}
\maketitle
\end{frame}

\section{Introduction and Summary}

\begin{frame}
  \frametitle{Outline}
  \begin{itemize}
  \item Introduction
  \item Summary of Results
  \item The Series Representation
  \item Model Error Bounds
  \item A New Solution Algorithm
  \end{itemize}
\end{frame}

\subsection{Introduction}
\label{sec:introduction}


\begin{frame}
  \frametitle{Introduction}
\begin{itemize}
\item Model solutions characterized by :

\begin{gather}
\eqnFuncSys \intertext{ where, each }
\eqnFuncSysI{i}\equiv \eqnFuncSysIExpl{i} \label{eqnGates}
\end{gather}
\item Boolean valued gate, $\gate_i$. 
\item model equations, $\eqnFunc_i$,  
\item ``Auxiliary variables'' if necessary
\item exhaustive and mutually exclusive determining the solution  $x_t\tArg$
\end{itemize}
\end{frame}

\subsection{Summary of Results}
\label{sec:summary-results}




\begin{frame}
  \frametitle{Model Error Bounds}
  \begin{itemize}
  \item Exact (typically unknown) solution $x^\star_t=g^\star(x_{t-1},\epsilon_t)$ 
  \item Proposed solution  $x^p_t=g^p(x_{t-1},\epsilon_t), \,\,G^p(x)\equiv \expct{g^p(x_{t-1},\epsilon_t)}$ 
 \item Euler errors $\eulerE_i\tArg \equiv  \eqnFunc(x_{t-1},g^p(x_{t-1},\epsilon),G^p(g^p(x_{t-1},\epsilon)),\epsilon)$
 \item We can easily compute matrices  $F, \phi $ such that 
 {\small   \begin{gather*}
 \someNorm{ x^\star_{t}(x_{t-1},\epsilon) -	 x^p_{t}(x_{t-1},\epsilon)} \le 
 \max_{\{i,x_{t-1},\epsilon\}} \someNorm{(I-F)^{-1} \phi \eulerE_i\tArg }
    \end{gather*}}
  \end{itemize}
\end{frame}
\begin{frame}
  \frametitle{Alternative Error Bounds}
  \begin{itemize}
  \item \cite{judd2017lower}
  \item \cite{peralta-alva14}
  \item \cite{santos2005accuracy}
  \item \cite{Santos2000accuracy}
  \end{itemize}
\end{frame}




\begin{frame}
  \frametitle{A New Algorithm}
 {\small  
\begin{itemize}
\item Using a formula from \citep{anderson10},
 \item Any bounded time invariant discrete map can be written as
    \begin{gather*}
      	 x\tArg =B x_{t-1}+ \phi \psi_\epsilon\epsilon + (I - F)^{-1} \phi \psi_c + \sum_{\sForSum=0}^\infty F^\sForSum \phi \ZWOarg(\expct{x_{t+\nu}})\\ \intertext{ so that}
\expct{ x_{t+1}\tArg} =B x\tArg  + (I - F)^{-1} \phi \psi_c+ \sum_{\sForSum =1}^\infty F^{\sForSum-1} \phi \ZWOarg(\expct{x_{t+\nu}}) 
    \end{gather*}
   \item If these functions satisfy the model equations you have a solution
   \item If not, use the $\eulerE_i \text{ and } \eqnFuncSys $ information to improve the solution
  \end{itemize}
}
\end{frame}

\begin{frame}
  \frametitle{Alternative Approaches}
  \begin{itemize}
\item \cite{Judd2014}
\item \cite{juddGSSA2011}
\item \cite{holden15:_exist_dsge}
\item  \cite{guerrieri15:_occbin}
  \end{itemize}
\end{frame}

\begin{frame}
  \frametitle{Advantages of the Series Formulation}
  \begin{itemize}
  \item Builds upon existing approaches
    \begin{itemize}
    \item Uses the an-isotropic Smolyak Method and adaptive paralletope method
    \end{itemize}
  \item Precomputes integrals
  \item Polynomial interpolation highly parallelizable
  \item very general functional form
  \end{itemize}
\end{frame}





\section{A New Series Representation For  Bounded Time Series}
\label{sec:newseries}

\subsection{Linear Reference Models and a Formula for  ``Anticipated Shocks''}
\label{sec:linref}



\subsection{ An  ``Arbitrary'' Linear Model and  Bounded Time Series}
\label{sec:almostarbitrary}



\begin{frame}
  \frametitle{Linear Reference Model}

  \begin{itemize}
  \item linear homogeneous system with  unique stable solution,  

\begin{gather}
  	 H_{-1} x_{t-1} + H_0 x_t + H_1 x_{t+1}=0\label{hSystem}
\end{gather}
\item inhomogeneous solutions 
\begin{gather}
	 H_{-1} x_{t-1} + H_0 x_t + H_1 x_{t+1}=\psi_\epsilon \epsilon +\psi_{c}
\intertext{ can be computed as}
x_t=B x_{t-1} + \phi \psi_\epsilon \epsilon + (I - F)^{-1} \phi \psi_c
\intertext{where}
\phi= (H_0 +H_1 B)^{-1}  \text{ and } \,\,F=-\phi H_1 
\end{gather}

\item Define $\linMod \equiv \linModMats$.
  \end{itemize}

\end{frame}
\begin{frame}
  \frametitle{Fully Anticipated Shocks}
{\small
\begin{theorem}
Arbitrary but bounded path
 \begin{gather*}
   \xWOarg_t \in{R^L}\,\text{ with }\,\infNorm{\xWOarg_t}  \le \bar{\mathcal{X}}\,\,\,\,\,\forall t> 0 
\intertext{ define  $  z_{t} \equiv H_{-1} \xWOarg_{t-1} +  H_0 \xWOarg_{t} +  H_1 \xWOarg_{t+1}   $ then }
	  \xWOarg_{t} =B x_{t-1}+ \phi \psi_\epsilon\epsilon + (I - F)^{-1} \phi \psi_c + \sum_{\sForSum=0}^\infty F^s \phi z_{t+\sForSum} \\
	  \xWOarg_{t+k+1} =B \xWOarg_{t+k}  + (I - F)^{-1} \phi \psi_c+ \sum_{\sForSum =0}^\infty F^\sForSum \phi z_{t+k+\sForSum+1} \,\,\,\,\,\forall t,k \ge  0.
	 \end{gather*}
\end{theorem}
}
\end{frame}

\subsection{A Numerical Example}
\label{sec:numerical-exampley}


\begin{frame}

  \frametitle{A Numerical Example}


\begin{gather}
  \begin{bmatrix}
H_{-1}&H_{0}&H_{1} 
  \end{bmatrix}=
\vcenter{\hbox{\includegraphics{refHmat.pdf}}}\intertext{with $\psi_c=\psi_\epsilon=0, \,\,  \psi_z=I$.}
  B=
\vcenter{\hbox{\includegraphics{refBmat.pdf}}}\\
\phi=
\vcenter{\hbox{\includegraphics{refPhimat.pdf}}}\\
F=
\vcenter{\hbox{\includegraphics{refFmat.pdf}}}
\end{gather} 

  
\end{frame}

\begin{frame}
  \frametitle{Some Time Series}


\begin{figure}
  \centering
\begin{gather}
  x_{1,t}=\alpha D_\pi(t) \\
x_{2,t}=\beta (-1)^t\\
x_{3,t}=\epsilon_t 
\end{gather} 
\includegraphics[width=1in]{piPath.pdf}
\includegraphics[width=1in]{oscillPath.pdf}
\includegraphics[width=1in]{pseudoPath.pdf}
\includegraphics[width=1in]{theZs.pdf}  
  
  \caption{Arbitrary Bounded Time Series Paths and Corresponding $z_{i,t}$ values}\label{arbpaths}
\end{figure}



\end{frame}

\subsection{Assessing Errors}
\label{sec:assessing-errors}


\begin{frame}
  \frametitle{Truncation Error}
 	 \begin{gather}
 	 \xWOargK_t \equiv B x_{t-1}+ \phi \psi_\epsilon\epsilon  + (I - F)^{-1} \phi \psi_c + \sum_{s=0}^k F^s \phi z_{t}\label{theTruncSeries}
 \end{gather}
We can bound the  series approximation truncation errors.
Since
    \begin{gather}
      \label{eq:1}
\sum_{s=k+1}^{\infty} F^s \phi \psi_z = (I -F)^{-1} F^{k+1}\phi \psi_z       \\
\infNorm{\xWarg-\xWargK} \le\\ \infNorm{(I -F)^{-1} F^{k+1}\phi \psi_z} \left ( \infNorm{H_{-1} }+ \infNorm{H_{0} }+ \infNorm{H_{1} } \right )\bar{\mathcal{X}}
    \end{gather}

\end{frame}
\begin{frame}
  \frametitle{Truncation Error}


\begin{figure}
  \centering


\includegraphics[width=3in]{arbTruncErr.pdf}  
  \caption{$x_t$ Error Bound Versus Actual Error} \label{figArbTrunc}

\end{figure}

\end{frame}

\begin{frame}
  \frametitle{Path Error}
One could consider approximating $\mathcal{X}_t$ using
 	 \begin{gather}
 	 \xWOargK_t \equiv B x_{t-1}+ \phi \psi_\epsilon\epsilon  + (I - F)^{-1} \phi \psi_c + \sum_{s=0}^\infty F^s \phi (z_{t}+\Delta z_{t})\label{theDeltaSeries}
 \end{gather}
We can bound the  series approximation  errors by using the largest $\Delta z_t$ in the formula. 

    \begin{gather}
\infNorm{\xWarg-\xWargK} \le \infNorm{(I -F)^{-1} \phi \psi_z}  \infNorm{\Delta z_t } \label{pathErr}
    \end{gather}

\end{frame}


\section{Nonlinear Dynamic Stochastic Time Invariant Maps}
\label{sec:extToMaps}



\subsection{Application to Time Invariant Maps}


\begin{frame}
  \frametitle{Time Invariant Maps}


  \begin{itemize}
\item  Many dynamic stochastic models 
  \begin{itemize}
  \item  have solutions that fall in this class.
\item  generate  bounded time series paths 
  \end{itemize}
  \end{itemize}



\subsection{An RBC Example}
\label{sec:rbcaux}
  We consider a model described in \cite{Maliar2005}\footnote{Here, we set their $\beta=1$ do not discuss quasi-geometric discounting or time-inconsistency.}
 \begin{gather*}
   \max\left \{  u(c_t) + E_t \sum_{t=0}^\infty  \delta^{t}u(c_{t+1})\right \}\\
c_t + k_t=(1-d)k_t + \theta_t f(k_t)\\
f(k_t)= k_t^\alpha\\
u(c)=\frac{c^{1-\eta}-1}{1-\eta}
 \end{gather*}
 \end{frame}

 \begin{frame}
   \frametitle{First Order Conditions}


The well known first order conditions for the model are

\begin{tcolorbox}[ams gather]
\frac{1}{c_t^{\eta}}=\alpha \delta k_{t}^{\alpha-1} E_t \left (\frac{\theta_{t}}{c_{t+1}^\eta} \right ) \\
c_t + k_t=\theta_{t-1}k_{t-1}^\alpha \\
 \theta_t =\theta_{t-1}^\rho e^{\epsilon_t}\label{rbcSys}
 \end{tcolorbox}

\label{sec:rbcexample}

When $\eta=d=1$, we have
{\small
 \begin{gather*}
   \begin{split}
\frac{1}{c_t}=\alpha \delta k_{t}^{\alpha-1} E_t \left (\frac{\theta_{t}}{c_{t+1}} \right ) \\
c_t + k_t=\theta_{t-1}k_{t-1}^\alpha \\
\theta_t =\theta_{t-1}^\rho e^{\epsilon_t}   \end{split} \text{   leading to a closed form solution}
\begin{split}
c_t=  (1-\alpha \delta) \theta_{t} k_{t-1}^\alpha\\
  k_{t}= \alpha \delta \theta_{t} k_{t-1}^\alpha.\label{soln}\\
\theta_t =\theta_{t-1}^\rho e^{\epsilon_t}.\end{split}
\end{gather*}
}
\end{frame}

\begin{frame}

For mean zero iid $\epsilon_t$ we can easily 
compute the conditional expectation of the model variables for any given $\theta_{t-1},k_{t-1}$
\begin{gather*}
  E_t(c_t|\theta_{t-1},k_{t-1})=(1-\alpha\delta)k_{t-1}^\alpha e^{\frac{\sigma^2}{2}}\theta_{t-1}^\rho\\
  E_t(k_t|\theta_{t-1},k_{t-1})=\alpha\delta k_{t-1}^\alpha e^{\frac{\sigma^2}{2}}\theta_{t-1}^\rho\\
  E_t(\theta_t|\theta_{t-1},k_{t-1})=e^{\frac{\sigma^2}{2}}\theta_{t-1}^\rho
\end{gather*}


By applying the law of iterated expectations, we can compute conditional expected solution paths forward from any initial values $x_{t-1}$
and realization of $\epsilon_t$.
As a result, we can use the family of conditional expectations
along with the contrived reference model to recover an 
approximation for equation \refeq{soln} along with error bounds.
The series representation provides a weighted sum of $z_t\tArg$ functions that give us
an approximation for the known solution.
Note that the reference model is deterministic and the $z_t\tArg$ functions account for the stochastic nature of the model.
\end{frame}

\begin{frame}
  


For any given values of $k_{t-1},\theta_{t-1}, \epsilon_t$ the model solution and conditional expectations path produces paths for $z_{1t}\tArg, z_{2t}\tArg, z_{3t}\tArg$

\begin{gather*}
  \begin{bmatrix}
c_t(k_{t-1},\theta_{t-1}, \epsilon_t)\\
k_t(k_{t-1},\theta_{t-1}, \epsilon_t)\\
\theta_t(k_{t-1},\theta_{t-1}, \epsilon_t)
  \end{bmatrix} \rightarrow
  \begin{bmatrix}
  z_{1t}(k_{t-1},\theta_{t-1}, \epsilon_t)\\
  z_{2t}(k_{t-1},\theta_{t-1}, \epsilon_t)\\
  z_{3t}(k_{t-1},\theta_{t-1}, \epsilon_t) 
  \end{bmatrix}\equiv z(k_{t-1},\theta_{t-1}, \epsilon_t)\intertext{where}
  \begin{bmatrix}
c_t(k_{t-1},\theta_{t-1}, \epsilon_t)\\
k_t(k_{t-1},\theta_{t-1}, \epsilon_t)\\
\theta_t(k_{t-1},\theta_{t-1}, \epsilon_t)
  \end{bmatrix}  =
B   \begin{bmatrix}
c_{t-1}\\
k_{t-1}\\
\theta_{t-1}
  \end{bmatrix}  + \phi \psi_\epsilon\epsilon_t + (I - F)^{-1} \phi \psi_c + \sum_{\sForSum=0}^\infty F^s \phi z_{t+\sForSum}(k_{t-1},\theta_{t-1}, \epsilon_t) 
\end{gather*}
% \footnote{
% We need not  make these adjustments for the steady state,
% but doing so economizes on the number of terms 
% required for a given level of approximation
% accuracy.}
\end{frame}
\begin{frame}
  



For example, using the following parameter values and using the arbitrary linear reference model we can generate a series representation for the model solutions.

\begin{gather}\label{rbcparams}
\vcenter{\hbox{\includegraphics{../../paperProduction/occBind/docs/RBCParamSubs.pdf}}} \,\, \text{ we have } \,\,
  \begin{bmatrix}
    c_{ss}\\k_{ss} \\ \theta_{ss} 
  \end{bmatrix}=
\left [ \vcenter{\hbox{\includegraphics{RBCSSVal.pdf}}}\right ]
\end{gather}

With 
\begin{gather}\label{theInits}
  \begin{bmatrix}
 k_{t-1}\\\theta_{t-1}\\\epsilon_t 
  \end{bmatrix}=
\left [ \vcenter{\hbox{\includegraphics{anXEps.pdf}}}\right ]
\end{gather}

\end{frame}
\begin{frame}
  

\begin{figure}
  \centering
\includegraphics[width=2.5in]{simprbcvals.pdf}  
\includegraphics[width=2.5in]{simprbczvals.pdf}  
  \caption{model variable values and z values}
  \label{rbcpaths}
\end{figure}

The left panel of Figure \refeq{rbcpaths} shows the paths, from top to bottom, of $\theta_t, c_t, \text{and} k_t$ from the initial values given in Equation \refeq{theInits}.  The right panel shows the paths for the $z_t$ variables associated with the linear reference model. The orange line corresponds to $z_{1t}\tArg$,
the blue line corresponds to $z_{2t}\tArg$ and the green line corresponds to $z_{3t}\tArg$.
\end{frame}

\begin{frame}
  

Figure \ref{rbcTrunc} shows the impact that truncating the series has 
on the initial value of the state variables.   The bound, $B_n$, shown in red 
again is very pessimistic compared to the actual, $Z_n$, shown in blue.
 But even with an arbitrarily chosen linear model, ultimately the series approximation provides an accurate value for the initial state vector.  

\begin{figure}
  \centering
\includegraphics[width=2.7in]{simpArbBoundsVActual.pdf}  
  \caption{RBC Truncation Error Bound Versus Actual}
  \label{rbcTrunc}
\end{figure}
\end{frame}

\begin{frame}
  

\begin{figure}
  \centering
\includegraphics[width=2.7in]{simpBoundsVActual.pdf}  
  \caption{RBC Truncation Error Bound Versus Actual}
  \label{rbcTruncSimp}
\end{figure}



Figure \ref{rbcTruncSimp} shows that using a 
linearization that better tracks the 
nonlinear model paths, improves the approximation $Z_n$, shown in blue  and the error bound $B_n$ shown in red.
Using the linearization of the RBC model produces a tighter but still pessimistic bound on the errors for the initial state vector.
The first few terms make most of the difference in approximating the value of the state variables.


\end{frame}

For convenience of notation in what follows, 
we will focus on models built up from components of the form
\begin{gather}
  h_i(x_{t-1},x_{t},x_{t+1},\epsilon_t)=h^{det}_{io}(x_{t-1},x_{t},\epsilon_t)+\sum_{j=1}^{p_i} [h^{det}_{ij}(x_{t-1},x_{t},\epsilon_t)h^{nondet}_{ij}(x_{t+1})]=0
\end{gather}
This is a very broad class of models including most widely used
macroeconomics models.

For example, the Euler equations for the  neoclassical growth  model 
\label{sec:simple-rbc-model-ext} can be written as
\begin{gather}
h_{10^{det}}(\cdot)=\frac{1}{c_t^\eta},\,\,
h_{11}^{det}()=\alpha \delta k_{t}^{\alpha-1} ,\,\,
h_{11}^{nondet}(\cdot)=E_t \left (\frac{\theta_{t+1}}{c_{t+1}^\eta} \right )\\
h_{20}^{det}(\cdot)=c_t + k_t-\theta_tk_{t-1}^\alpha,\,\,
h_{21}^{det}(\cdot)=0\\
h_{30}^{det}(\cdot)=\ln \theta_t -(\rho \ln \theta_{t-1} + \epsilon_t),\,\,
h_{31}^{det}(\cdot)=0
\end{gather}
Since we would otherwise  need to compute 
the conditional expectation of nonlinear expressions,  
this setup will make it possible for us to use auxiliary
variables to correctly compute the required expected values.

It is worth noting that since we will be working with models where expectations are computed at time t, with  $\epsilon_t$  known,  the only stochastic components are those with time subscripts greater than $t$. 





We now construct our linear reference model by increasing its dimension by one , $\linMod$, to accommodate 
 augmenting the RBC model with the equation
\begin{tcolorbox}[ams gather]
  \rcpC_t=\frac{\eta}{c_t}
\end{tcolorbox}
substituting $\rcpC_{t+1}$ for $\frac{1}{c_{t+1}}$ in the first equation and 
 linearizing the RBC model about the ergodic mean
given in \refeq{rbcparams}
{\small
\begin{gather}
  \begin{bmatrix}
H_{-1}&H_{0}&H_{1} 
  \end{bmatrix}=\\
\vcenter{\hbox{\includegraphics{RBCHmatSymb.pdf}}} \label{rbcLinSys}
\intertext{with}
\psi_\epsilon=
\begin{bmatrix}
  0\\0\\1\\0
\end{bmatrix}, \psi_z=I
\end{gather}%(\footnote{generated by AMAPaperCalcs.mth {RBCHmatSymb.pdf}})
}
These coefficients  produce a unique stable linear solution.

\begin{gather}
  B=
\vcenter{\hbox{\includegraphics{RBCBmatSymb.pdf}}},
\phi=
\vcenter{\hbox{\includegraphics{RBCPhimatSymb.pdf}}}\\
F=
\vcenter{\hbox{\includegraphics{RBCFmatSymb.pdf}}}\\
\psi_c=
\vcenter{\hbox{\includegraphics{RBCHSum.pdf}}}
\vcenter{\hbox{\includegraphics{RBCSS.pdf}}}=\vcenter{\hbox{\includegraphics{RBCPsissSymb.pdf}}}
\end{gather}


\section{A Model Error Bound}
\label{sec:bound}


\section{A New Solution Algorithm}
\label{sec:new-solut-algor}


\section{Conclusions}
\label{sec:conclusions}


\begin{frame}
  \frametitle{Next Steps}
\begin{itemize}
\item Support Vector Machine Regression (SVMR)
\item Large Model Implementation
\item Improve Error Bounds
\item Dynare Interface
\end{itemize}

\end{frame}



\bibliographystyle{plainnat}
\bibliography{anderson,files}

\end{document}
